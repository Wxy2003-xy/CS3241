\documentclass[a4paper]{article}
\usepackage[margin=0.4in]{geometry}
\usepackage[T1]{fontenc}
\usepackage[utf8]{inputenc}
\usepackage{lmodern}
\usepackage{amsmath}
\usepackage{amssymb}
\usepackage{geometry}
\usepackage{enumerate}
\usepackage{xcolor}
\usepackage{graphicx}
\usepackage{amsthm}
\newtheorem{theorem}{Theorem}[section]
\newtheorem{lemma}[theorem]{Lemma}
\newtheorem{corollary}[theorem]{Corollary}
\newtheorem{definition}[theorem]{Definition}
\usepackage{listings}
\usepackage{authblk}
\usepackage{titling}
\usepackage{zlmtt}
\usepackage{fancyvrb}
\usepackage[hidelinks]{hyperref}
\setlength{\droptitle}{-2em}
\lstset{frame=tb,
  language=C,
  aboveskip=3mm,
  belowskip=3mm,
  showstringspaces=false,   
  columns=flexible,
  basicstyle={\small\ttfamily},
  numbers=none,
  numberstyle=\tiny\color{gray},
  keywordstyle=\color{blue},
  commentstyle=\color{brown},
  stringstyle=\color{orange},
  breaklines=true,
  breakatwhitespace=true,
  tabsize=3
}
\begin{document}
\title{CS3241 tut 1}
\author{
  Wang Xiyu
}
\maketitle
1.Differences between a point and a vector: 
When translate a point the vector points to it from the origin of the reference frame changes, while a vector when translated remains the direction.

2. (3,1,10)t -(4,1,2)-> (7,2,12)t

3. (-4,-1,-2)

4. (3,1,10,1)

5. matrix T = 
[[1 0 0 4],
[0 1 0 1],
[0 0 1 2],
[0 0 0 1]]

express T (3,1,10,1)t = (7,2,12,1)t

6. y = -x + 3

7.Solve equation y in x = c.
10 = -x + 3 -> x = -7

8. L(t) = (1 - t)a + tb, 

12. purple

14. not possible to violate:
simple
convex 
planar


\section{tut 2}

1. each loop iteration renders a frame/update
2. initialize -> register call back functions -> enter glutMainLoop -> wait for event 

3. call when want to trigger event
4. raster: easy to process, vector: not resolution dependent
5. double buffer: 

\end{document}
